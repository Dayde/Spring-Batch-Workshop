\section{Flat file reading}

\begin{frame}
 \begin{itemize}
  \item Problem: reading lines from a flat file and sending them to another source (e.g. database)
  \item Solution: using the \code{FlatFileItemReader}
 \end{itemize}
\end{frame}

\begin{frame}
 \frametitle{Spring Batch's support for flat file reading}
 \begin{itemize}
  \item Spring Batch has built-in support for flat files
  \begin{itemize}
    \item Through the \code{FlatFileItemReader} for reading
  \end{itemize}  
  \item The \code{FlatFileItemReader} handles I/O
  \item 2 main steps: 
  \begin{itemize}
    \item Configuring the \code{FlatFileItemReader}
    \item Providing a line-to-object mapping strategy
  \end{itemize}
 \end{itemize}
\end{frame}

\begin{frame}[fragile]
 \frametitle{The usual suspects}
 \begin{textcode}
Susy,Hauerstock,2010-03-04
De Anna,Raghunath,2010-03-04
Kiam,Whitehurst,2010-03-04
Alecia,Van Holst,2010-03-04
Hing,Senecal,2010-03-04
\end{textcode}

\begin{center}
\begin{picture}(0,0)
\put(0,15){\vector(0,-1){15}} 
\end{picture}
\end{center}

\begin{javacode}
public class Contact {

  private Long id;
  private String firstname,lastname;
  private Date birth;
  
  (...)
}
\end{javacode}

\end{frame}

\begin{frame}
 \frametitle{What do we need to read a flat file?}
 \begin{itemize}
  \item How to tokenize a line
  \item How to map the line with a Java object
  \item Where to find the file to read
 \end{itemize}
\end{frame}


\begin{frame}[fragile]
 \frametitle{The \code{FlatFileItemReader} configuration}

\begin{xmlcode}
<bean id="reader"
      class="o.s.b.item.file.FlatFileItemReader">
  <property name="lineMapper">
    <bean class="o.s.b.item.file.mapping.DefaultLineMapper">
      <property name="lineTokenizer">
        <bean class="o.s.b.item.file.transform.DelimitedLineTokenizer">
          <property name="names" value="firstname,lastname,birth" />
        </bean>
      </property>
      <property name="fieldSetMapper">
        <bean class="c.z.workshop.springbatch.ContactFieldSetMapper" />
      </property>
    </bean>
  </property>
  <property name="resource" value="classpath:contacts.txt" />
</bean>
\end{xmlcode}

\end{frame}

\begin{frame}[fragile]
 \frametitle{The \code{FlatFileItemReader} declaration}

\begin{xmlcode}
<bean id="reader"
      class="o.s.b.item.file.FlatFileItemReader">













</bean>
\end{xmlcode}

\end{frame}

\begin{frame}[fragile]
 \frametitle{How to tokenize a line}

\begin{xmlcode}
<bean id="reader"
      class="o.s.b.item.file.FlatFileItemReader">
  <property name="lineMapper">
    <bean class="o.s.b.item.file.mapping.DefaultLineMapper">
      <property name="lineTokenizer">
        <bean class="o.s.b.item.file.transform.DelimitedLineTokenizer">
          <property name="names" value="firstname,lastname,birth" />
        </bean>
      </property>



    </bean>
  </property>

</bean>
\end{xmlcode}
\end{frame}

\begin{frame}[fragile]
 \frametitle{How to map the line with a Java object}


\begin{xmlcode}
<bean id="reader"
      class="o.s.b.item.file.FlatFileItemReader">
  <property name="lineMapper">
    <bean class="o.s.b.item.file.mapping.DefaultLineMapper">





      <property name="fieldSetMapper">
        <bean class="c.z.workshop.springbatch.ContactFieldSetMapper" />
      </property>
    </bean>
  </property>

</bean>
\end{xmlcode}

\end{frame}

\begin{frame}[fragile]
 \frametitle{Where to find the file to read}

\begin{xmlcode}
<bean id="reader"
      class="o.s.b.item.file.FlatFileItemReader">












  <property name="resource" value="classpath:contacts.txt" />
</bean>
\end{xmlcode}

\end{frame}

\begin{frame}[fragile]
 \frametitle{The \code{FlatFileItemReader} configuration}

\begin{xmlcode}
<bean id="reader"
      class="o.s.b.item.file.FlatFileItemReader">
  <property name="lineMapper">
    <bean class="o.s.b.item.file.mapping.DefaultLineMapper">
      <property name="lineTokenizer">
        <bean class="o.s.b.item.file.transform.DelimitedLineTokenizer">
          <property name="names" value="firstname,lastname,birth" />
        </bean>
      </property>
      <property name="fieldSetMapper">
        <bean class="c.z.workshop.springbatch.ContactFieldSetMapper" />
      </property>
    </bean>
  </property>
  <property name="resource" value="classpath:contacts.txt" />
</bean>
\end{xmlcode}

\end{frame}

\begin{frame}
 \frametitle{The line-to-object mapping strategy}
 \begin{itemize}
  \item A \code{FieldSetMapper} to map a line with an object
  \item More about business logic, so typically implemented by developer
  \item Spring Batch provides straightforward implementations
 \end{itemize}
\end{frame}

\begin{frame}[fragile]
 \frametitle{Custom \code{FieldSetMapper} implementation}

\begin{javacode}
package com.zenika.workshop.springbatch;

import org.springframework.batch.item.file.mapping.FieldSetMapper;
import org.springframework.batch.item.file.transform.FieldSet;
import org.springframework.validation.BindException;

public class ContactFieldSetMapper implements FieldSetMapper<Contact> {

  @Override
  public Contact mapFieldSet(FieldSet fieldSet) throws BindException {
    return new Contact(
      fieldSet.readString("firstname"),
      fieldSet.readString("lastname"), 
      fieldSet.readDate("birth","yyyy-MM-dd")
    );
  }
}
\end{javacode}

\end{frame}


\begin{frame}
 \frametitle{Going further...}
 \begin{itemize}
  \item \code{FlatFileItemWriter} to write flat file
  \item Fixed-length format (different tokenizer)
  \item Skipping badly formatted lines
 \end{itemize}
\end{frame}
