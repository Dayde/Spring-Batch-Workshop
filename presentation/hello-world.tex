\section{Hello World}

\begin{frame}
 \begin{itemize}
  \item Problem: getting started with Spring Batch
  \item Solution: writing a simple ``Hello World'' job
 \end{itemize}
\end{frame}

\begin{frame}
 \frametitle{Structure of a job}
 \begin{itemize}
  \item A Spring Batch job is made of steps
  \item The Hello World job has one step
  \item The processing is implemented in a \code{Tasklet}
 \end{itemize}
\end{frame}

\begin{frame}[fragile]
\frametitle{The Hello World \code{Tasklet}}
\lstset{language=Java}
\begin{lstlisting}
public class HelloWorldTasklet implements Tasklet {

  @Override
  public RepeatStatus execute(
      StepContribution contribution,
      ChunkContext chunkContext) throws Exception {
    System.out.println("Hello world!");
    return RepeatStatus.FINISHED;
  }

}
\end{lstlisting}
\end{frame}

\begin{frame}[fragile]
\frametitle{The configuration of the Hello World job}
\lstset{language=XML}
\begin{lstlisting}
<batch:job id="helloWorldJob">
  <batch:step id="helloWorldStep">
    <batch:tasklet>
      <bean class="com.zenika.workshop.springbatch.HelloWorldTasklet" />
    </batch:tasklet>
  </batch:step>
</batch:job>
\end{lstlisting}

\begin{itemize}
 \item Notice the \code{batch} namespace
\end{itemize}

\end{frame}

\begin{frame}[fragile]
\frametitle{Spring Batch needs some infrastructure beans}
\begin{itemize}
 \item Let's use the typical test configuration
\end{itemize}

\lstset{language=XML}
\begin{lstlisting}
<bean id="transactionManager"
      class="o.s.b.support.transaction.ResourcelessTransactionManager" />

<bean id="jobRepository"   
      class="o.s.b.core.repository.support.MapJobRepositoryFactoryBean" />

<bean id="jobLauncher"
      class="o.s.b.core.launch.support.SimpleJobLauncher">
  <property name="jobRepository" ref="jobRepository" />
</bean>
\end{lstlisting}
\end{frame}

\begin{frame}[fragile]
\frametitle{Running the test in a JUnit test}

\lstset{language=Java}
\begin{lstlisting}
@RunWith(SpringJUnit4ClassRunner.class)
@ContextConfiguration("/hello-world-job.xml")
public class HelloWorldJobTest {

  @Autowired
  private Job job;

  @Autowired
  private JobLauncher jobLauncher;

  @Test public void helloWorld() throws Exception {
    JobExecution execution = jobLauncher.run(job, new JobParameters());
    assertEquals(ExitStatus.COMPLETED, execution.getExitStatus());
  }
}
\end{lstlisting}
\end{frame}
