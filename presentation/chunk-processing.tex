\section{Chunk processing}

\begin{frame}
 \begin{itemize}
  \item Problem: processing large amounts of data efficiently
  \item Solution: using chunk processing
 \end{itemize}
\end{frame}

\begin{frame}
 \frametitle{What is chunk processing?}
 \begin{itemize}
  \item Batch jobs often read, process, and write items
  \item e.g.
  \begin{itemize}
    \item Reading items from a file
    \item Then processing (converting) items
    \item Writing items to a database  
  \end{itemize}
  \item Spring Batch calls this ``chunk processing''
  \item a chunk = a set of items
 \end{itemize}
\end{frame}


\begin{frame}
 \frametitle{Chunk processing with Spring Batch}
 \begin{itemize}
  \item Spring Batch
  \begin{itemize}
    \item handles the iteration logic
    \item uses a transaction for each chunk
    \item lets you choose the chunk size
    \item defines interfaces for each part of the processing
  \end{itemize}
 \end{itemize}
\end{frame}


\begin{frame}[fragile]
\frametitle{The reading phase}
\begin{itemize}
 \item Spring Batch creates chunks of items by calling \code{read()}
 \item Reading ends when \code{read()} returns \code{null}
\end{itemize}

\lstset{language=Java}
\begin{lstlisting}
public interface ItemReader<T> {

  T read() throws Exception, UnexpectedInputException, 
                  ParseException, NonTransientResourceException;

}
\end{lstlisting}
\end{frame}

\begin{frame}[fragile]
\frametitle{The processing phase}
\begin{itemize}
 \item Once a chunk is created, items are sent to the processor
 \item Optional
\end{itemize}

\lstset{language=Java}
\begin{lstlisting}
public interface ItemProcessor<I, O> {

  O process(I item) throws Exception;

}
\end{lstlisting}
\end{frame}

\begin{frame}[fragile]
\frametitle{The writing phase}
\begin{itemize}
 \item Receives all the items of the chunk
 \item Allows for batch update (more efficient)
\end{itemize}

\lstset{language=Java}
\begin{lstlisting}
public interface ItemWriter<T> {

  void write(List<? extends T> items) throws Exception;

}
\end{lstlisting}
\end{frame}

\begin{frame}
 \frametitle{An example}
 \begin{itemize}
  \item Let's implement a (too?) simple chunk-oriented step!  
 \end{itemize}
\end{frame}

\begin{frame}[fragile]
\frametitle{The \code{ItemReader}}
\lstset{language=Java}
\begin{lstlisting}
public class StringItemReader implements ItemReader<String> {

  private List<String> list;

  public StringItemReader() {
    this.list = new ArrayList<String>(Arrays.asList(
      "1","2","3","4","5","6","7")
    );
  }

  @Override
  public String read() throws Exception, UnexpectedInputException,
                         ParseException, NonTransientResourceException {
    return !list.isEmpty() ? list.remove(0) : null;
  }
}
\end{lstlisting}
\end{frame}

\begin{frame}[fragile]
\frametitle{The \code{ItemProcessor}}
\lstset{language=Java}
\begin{lstlisting}
public class StringItemProcessor implements ItemProcessor<String, String> {

  @Override
  public String process(String item) throws Exception {
    return "*** "+item+" ***";
  }

}
\end{lstlisting}
\end{frame}

\begin{frame}[fragile]
\frametitle{The \code{ItemWriter}}
\lstset{language=Java}
\begin{lstlisting}
public class StringItemWriter implements ItemWriter<String> {

  private static final Logger LOGGER =
    LoggerFactory.getLogger(StringItemWriter.class);

  @Override
  public void write(List<? extends String> items) throws Exception {
    for(String item : items) {
      LOGGER.info("writing "+item);
    }
  }

}
\end{lstlisting}
\end{frame}

\begin{frame}[fragile]
\frametitle{Configuring the job}
\lstset{language=XML}
\begin{lstlisting}
<batch:job id="chunkProcessingJob">
  <batch:step id="chunkProcessingStep">
    <batch:tasklet>
      <batch:chunk reader="reader" processor="processor" writer="writer"
                   commit-interval="3"
      />
    </batch:tasklet>
  </batch:step>
</batch:job>

<bean id="reader" class="com.zenika.workshop.springbatch.StringItemReader" />

<bean id="processor" 
      class="com.zenika.workshop.springbatch.StringItemProcessor" />

<bean id="writer" class="com.zenika.workshop.springbatch.StringItemWriter" />
\end{lstlisting}
\end{frame}

\begin{frame}
 \frametitle{Considerations}
 \begin{itemize}
  \item Do I always need to write my \code{ItemReader}/\code{Processor}/\code{Writer}?
  \item No, Spring Batch provides ready-to-use components for common datastores
  \begin{itemize}
    \item Flat/XML files, databases, JMS, etc.
  \end{itemize}
  \item As an application developer, you
  \begin{itemize}
    \item Configure these components
    \item Provides some logic (e.g. mapping a line with a domain object)
  \end{itemize}
 \end{itemize}
\end{frame}

\begin{frame}
 \frametitle{Going further...}
 \begin{itemize}
  \item Reader/writer implementation for flat/XML files, database, JMS
  \item Skipping items when something goes wrong
  \item Listeners to react to the chunk processing  
 \end{itemize}
\end{frame}

